\documentclass[a4paper,10pt]{article} 
%\usepackage[style=nature,backend=biber]{biblatex}
\usepackage{graphicx}
\usepackage[font={it},labelfont={bf}]{caption}
\usepackage{times}
\usepackage{color}
%\usepackage{fixltx2e} % fixes placing of image positions
\usepackage{authblk}
\usepackage{amsmath}
\usepackage{courier}
\usepackage[round]{natbib}
\usepackage{hhline}

\definecolor{customhdrcolor}{rgb}{0.0,0.0,0.0}
%\definecolor{customcitecolor}{rgb}{0.0,0.25,0.25}
\definecolor{customcitecolor}{rgb}{0.0,0.5,0.75}
%\definecolor{customlinkcolor}{rgb}{0.0,0.0,1.0}
\definecolor{customlinkcolor}{rgb}{0.0,0.5,0.75}

\usepackage[colorlinks=true,linkcolor=customlinkcolor,urlcolor=customlinkcolor,citecolor=customcitecolor,pdftex]{hyperref}

\ifpdf\pdfinfo{/Title      (MWA paper proposal Offringa)
               /Author     (A. R. Offringa)
               /Keywords   (MWA, imaging)
        }
\else\usepackage{graphics}\fi

\setlength{\pdfpageheight}{\paperheight}
\setlength{\pdfpagewidth}{\paperwidth}

%\title{MWA Paper proposal:\\````WSClean: a fast implementation of a generic wide-field imager''}

%\author{A.~R.~Offringa}
%\author[1,2]{A.~R.~Offringa}
%\affil[1]{RSAA, Australian National University, Mt Stromlo Observatory, via Cotter Road, Weston, ACT 2611, Australia}
%\affil[2]{ARC Centre of Excellence for All-Sky Astrophysics (CAASTRO)}
\begin{document}


\label{firstpage}
\section*{MWA paper proposal: ``WSClean: a new implementation of a fast, generic wide-field imager''}
\begin{tabular}{ll}
Publication title: & WSClean: a new implementation of a fast, generic \\
                   & wide-field imager\\
                   & \\
Principle contact: & Dr. Andr\'e R. Offringa \\
                   & \\
Draft publication        & A.~R.~Offringa, B.~McKinley, N.~Hurley-Walker, \\
\hspace{5mm}author list: & F.~H.~Briggs, J. Rhee, MWA commissioning team \\
                         & and GLEAM members where appropriate, MWA \\
                         & builders list \\
                   & \\
Journal:           & Probably MNRAS\\
                   & \\
Intended date of submission\\
\hspace{5mm}to collaboration: & February 2014\\
\end{tabular}

\section*{Paper summary}
We present our new imager implementation, ``WSClean'', which we find to be an order of magnitude faster than existing implementations, as well as being capable of MWA full-sky imaging at full resolution.

Observations from non-coplanar interferometric radio telescopes that image large fractions of the sky at once can not be imaged with simple imagers that are based on a two-dimensional Fast Fourier Transformation (FFT). Instead, the imaging algorithm needs to account for the ``w-term'' during inversion, which is the term that describes the deviation of the array from a perfect plane. This deviation is amplified for telescopes with wide field of view, making this problem especially hard for low-frequency wide-field telescopes.

The most commonly used method to image observations from non-coplanar arrays is the w-projection algorithm \citep{wprojection-cornwell} that has been implemented in CASA. This is the fastest generic wide-field imager available, and CASA is the only software package (known to us) that implements the w-projection algorithm. Other packages rely on the faceting algorithm \citep{facetting-cornwell}, which is about an order of magnitude slower and is less accurate \citep{wprojection-cornwell}. The w-projection algorithm has thus been successful so far. However, the available implementations are unable to cope with the data rates from the new generation of wide-field observatories, which are producing data sets that are orders of magnitude larger than before. Examples of such telescopes include the Murchison Widefield Array (MWA), the upgraded Jansky Very Large Array (JVLA) and the Low-Frequency Array (LOFAR). At the MWA, we have seen that it can take up to tens of hours with the w-projection algorithm to image a two-minute MWA observation when the observed field is at low elevation. Moreover, the w-projection implementation in CASA can not handle images of 10 k $\times$ 10 k, which is desired for making full-sky images at the native resolution of the MWA. This will become an even bigger problem when the Square-Kilometre Array (SKA) begins its operation, and hence a considerable improvement in imaging techniques is required.

We present a new implementation of a generic wide-field imager that significantly improves upon the w-projection implementation. Currently it performs approximately an order of magnitude faster in all situations and even more for very large images, while it maintains equal accuracy and can handle MWA full-sky imaging. The implementation uses the ``w-stacking method'' for correcting the w-terms \citep{widefield-imaging-ska-cornwell} to obtain the increase in speed. Performance can be improved further by combination with another technique: the MWA telescope splits observations in data products of typically a few minutes, and the w-terms can be kept small by phasing an observation to the orthogonal of the array's best-fitting plane. This is somewhat similar to the w-snapshot algorithm proposed in \citet{widefield-imaging-ska-cornwell}. Hence, we can describe the imager as a hybrid between w-stacking and a variation of the w-snapshot algorithm. It also implements the Clean deconvolution procedure (\citealt{hogbom-clean} and various variations). We named the new imager ``WSClean'', as an abbreviation for ``W-Stacking Clean''.

After describing the new imaging implementation, we will test its performance using various MWA observations, including observations with varying elevation and fields (diffuse emission, point sources). GMRT data and -- if available -- LOFAR data will also be used. We will compare these results with the results from the w-projection implementation in CASA and to other non-w-projection algorithms, thereby focusing on accuracy and performance. The data sets that we currently envision to use are summarized in Table~\ref{tbl:observations}.

Once the paper is accepted for publication in a scientific journal, the WSClean source code will be publicly released.

\begin{figure*}[bh]
\begin{center}
\includegraphics[width=12cm]{EoR0-apparent.png}
\caption{A deep MWA image of EoR0 created from one hour of observing, imaged and deconvolved with WSClean in about an hour using 10 nodes on Fornax. Image has not been beam corrected. Noise level after correcting is 7 mJy. 200 sources were subtracted from the centre before imaging.}
\label{fig:stddev-spectrum}
\end{center}
\end{figure*}

\begin{table}
\caption{Data sets envisioned to be used in publication} \label{tbl:observations}
 \begin{tabular}{|c|l|p{6cm}|}
 \hhline{|=|=|=|}
 \textbf{Observatory} & \textbf{Description} & \textbf{Reasons} \\
 \textbf{+ date} & & \\
 \hhline{|=|=|=|}
  2013-05-02 & Decl -55          & $\bullet$ Test imaging at different frequencies\\
  MWA 128T   & GLEAM test data   & $\bullet$ Show computational performance \\
             &                   & \hspace{5mm} at elevation $\sim 60$\\
             &                   & $\bullet$ Full-sky imaging \\
 \hline
  2013-05-16 & Vela and Puppis A & $\bullet$ Test cleaning of diffuse structure \\
  MWA 128T   &                   & $\bullet$ Show computational performance \\
             &                   & \hspace{5mm} of cleaning\\
             &                   & $\bullet$ Pointed observations \\
 \hline
  2013-07-31 & EoR0              & $\bullet$ Test cleaning of point sources \\
  MWA 128T   &                   & $\bullet$ Test deep imaging \\
 \hline
  2008-2009  & ZCosmos           & $\bullet$ Test relevance to GMRT data \\
  GMRT       & (stacking)        & $\bullet$ Verify making image cubes \\
 \hline
  LOFAR      & To be determined  & $\bullet$ Test relevance to LOFAR data \\
             &                   & (only if data easily obtainable) \\
 \hline
             & Simulations       & $\bullet$ Validate photometry / astrometry \\
   ---       &                   & $\bullet$ Test computational performance \\
             &                   & \hspace{5mm} at very low elevation\\
 \hhline{|=|=|=|}
\end{tabular}
\end{table}

\label{lastpage}

\bibliographystyle{plainnat}
\bibliography{references}

\end{document}
