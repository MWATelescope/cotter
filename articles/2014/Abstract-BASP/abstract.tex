
% *** Authors should verify (and, if needed, correct) their LaTeX system  ***
% *** with the testflow diagnostic prior to trusting their LaTeX platform ***
% *** with production work. IEEE's font choices can trigger bugs that do  ***
% *** not appear when using other class files.                            ***
% The testflow support page is at:
% http://www.michaelshell.org/tex/testflow/



% Note that the a4paper option is mainly intended so that authors in
% countries using A4 can easily print to A4 and see how their papers will
% look in print - the typesetting of the document will not typically be
% affected with changes in paper size (but the bottom and side margins will).
% Use the testflow package mentioned above to verify correct handling of
% both paper sizes by the user's LaTeX system.
%
% Also note that the "draftcls" or "draftclsnofoot", not "draft", option
% should be used if it is desired that the figures are to be displayed in
% draft mode.
%
\documentclass[9pt,conference,a4paper]{IEEEtran}
% Add the compsoc option for Computer Society conferences.
%
% If IEEEtran.cls has not been installed into the LaTeX system files,
% manually specify the path to it like:
% \documentclass[conference]{../sty/IEEEtran}





% Some very useful LaTeX packages include:
% (uncomment the ones you want to load)


% correct bad hyphenation here
%\hyphenation{op-tical net-works semi-conduc-tor}


\begin{document}

\title{Radio interferometric imaging for the SKA and its pathfinders}

\author{%
\IEEEauthorblockN{
Andr\'e R. Offringa\IEEEauthorrefmark{1}}
%
\IEEEauthorblockA{\IEEEauthorrefmark{1} 
Research School of Astronomy \& Astrophysics,
Australian National University,
Weston Creek, 2611 ACT
Australia.}
%
} % end of author

\maketitle


\begin{abstract}
To make maps of the sky with a radio interferometer, the radio data has to be processed with an imaging algorithm. The SKA pathfinders show that this imaging step can be a considerable challenge. The imaging complexity scales with the number of elements, maximum baseline length, angular field of view, data volume and dynamic range. The SKA will be larger by orders of magnitude on all these aspects, and will therefore require new advances in solving the imaging problem. I will give an overview of the challenges involved, and summarize the currently available approaches.\end{abstract}

% Either use sections (but not subsections please) or add a 9pt space
% here.  If you use sections then please remove the 9pt space.
%\vspace*{9pt}
\section{Introduction}
Many algorithms are available for imaging radio data, that each have their own benefits and performance dependencies. In practice, many of these algorithms are combined, making a full imaging program very complex. In the following sections, I will describe the commonly used algorithms and their use for the SKA.

\section{$w$-term correction}
The first step in imaging is the mathematical inversion of the function that maps the sky brightness to the correlation products. The performance of this step often dominates the total imaging time. Arrays such as the SKA that will have a wide field of view and non-coplanar elements require expensive correction for the '$w$-terms', which describe the distance of an element to the plane. The fastests algorithms to correct for the $w$-terms are $w$-projection \cite{wprojection-cornwell} and $w$-stacking \cite{ska-memo-regridding-2011}. These algorithms have a squared and linear dependency on the size of the $w$-term respectively, although the faster $w$-stacking algorithm requires more memory \cite{offringa-wsclean-2014}. Further improvements can be made by combining either of these techniques with snapshot imaging. This decreases the size of the $w$-term, but requires regridding of the snapshot images before they can be integrated or deconvolved \cite{widefield-imaging-ska-cornwell}. This algorithm has recently been implemented in \textsc{wsclean}, and is useful for MWA imaging of very off-zenith or very large fields of view \cite{offringa-wsclean-2014}.

\section{Correction of directional effects}
Direction-dependent effects (DDEs) such as the ionosphere or a changing or heterogenous primary beam -- as will be the case for the SKA -- have to be corrected before images from different epochs are added together. It has been shown this can be corrected during the inversion step using the $a$-projection technique \cite{aprojection-2008} or, when the changes are uniform over the elements, by correcting snapshots in image space \cite{offringa-wsclean-2014}. For certain effects, a hybrid of these two can be used to reduce cost.

\section{Deconvolution}
An image created by inversion is convolved with the point-spread function (PSF) of the instrument, and therefore needs to be deconvolved to improve the dynamic range. For a wide-field array such as the SKA, the PSF varies spatially, which complicates this process. The Cotton-Schwab algorithm \cite{cotton-schwab-clean} corrects this: Minor iterations assume a constant PSF and produce a model for the image, while major iterations calculate and subtract the full effect of the changing PSF. The simplest algorithm for implementing minor iterations is the H\"ogbom algorithm \cite{hogbom-clean}. In more complex fields, multi-scale (MS) cleaning improves the deconvolution accuracy \cite{multiscale-clean-cornwell-2008}, but the application of \textsc{casa}'s MS algorithm on MWA data shows the algorithm is too slow for practical use with a wide-field array with so many elements. A fast multi-resolution multi-scale algorithm has recently been implemented in \textsc{wsclean} (Offringa et al., in prep.). Recently, deconvolution using compressed sensing (CS) have shown promising results \cite{wenger-compressed-sensing-2010}. CS improves the deconvolution accuracy over H\"ogbom cleaning, but its performance and robustness have not been extensively validated using real data.

Deconvolving multi-frequency synthesis (MFS) images is complicated further when a large bandwidth is imaged, making it necessary to account for spectral variation of sources. The Sault-Wieringa algorithm \cite{sault-wieringa-1994} takes spectral variation into account during deconvolution, and produces the MFS map as well as a spectral-index map. A different approach is taken by \textsc{wsclean}, which splits the bandwidth in subbands and deconvolves them joinedly. Further variation can be corrected during the major iterations.

\section{Parallellism \& data format}
In practical situations, a hybrid of these algorithms is often used, although different science goals will need different subsets of algorithms. Software packages such as \textsc{casa} and \textsc{wsclean} already implement many of these algorithms. Adaptions will be necessary for SKA's data volumes, such as parallellization over subband, short snapshot or image subset (mosaicing). The data will need to be available in such a way that each node can work on a subset of the data.

\DeclareRobustCommand{\TUSSEN}[3]{#3}

\bibliographystyle{IEEEtran}
\bibliography{bib}

\end{document}


