
% *** Authors should verify (and, if needed, correct) their LaTeX system  ***
% *** with the testflow diagnostic prior to trusting their LaTeX platform ***
% *** with production work. IEEE's font choices can trigger bugs that do  ***
% *** not appear when using other class files.                            ***
% The testflow support page is at:
% http://www.michaelshell.org/tex/testflow/



% Note that the a4paper option is mainly intended so that authors in
% countries using A4 can easily print to A4 and see how their papers will
% look in print - the typesetting of the document will not typically be
% affected with changes in paper size (but the bottom and side margins will).
% Use the testflow package mentioned above to verify correct handling of
% both paper sizes by the user's LaTeX system.
%
% Also note that the "draftcls" or "draftclsnofoot", not "draft", option
% should be used if it is desired that the figures are to be displayed in
% draft mode.
%
\documentclass[9pt,conference,a4paper]{IEEEtran}
% Add the compsoc option for Computer Society conferences.
%
% If IEEEtran.cls has not been installed into the LaTeX system files,
% manually specify the path to it like:
% \documentclass[conference]{../sty/IEEEtran}





% Some very useful LaTeX packages include:
% (uncomment the ones you want to load)


% correct bad hyphenation here
%\hyphenation{op-tical net-works semi-conduc-tor}


\begin{document}

\title{Radio interferometric imaging for the SKA and its pathfinders}

\author{%
\IEEEauthorblockN{
Andr\'e R. Offringa\IEEEauthorrefmark{1}}
%
\IEEEauthorblockA{\IEEEauthorrefmark{1} 
Research School of Astronomy \& Astrophysics,
Australian National University,
Weston Creek, 2611 ACT
Australia.}
%
} % end of author

\maketitle


\begin{abstract}
Please add a very brief abstract here (max 150 words).
\end{abstract}


% Either use sections (but not subsections please) or add a 9pt space
% here.  If you use sections then please remove the 9pt space.
%\vspace*{9pt}
\section{Introduction}
To make maps of the sky with a radio interferometer, the radio data has to be processed with an imaging algorithm, and the SKA pathfinders show that this imaging step is a considerable challenge \cite{awimager-2013,offringa-wsclean-2014}. The imaging complexity scales with the number of elements, maximum baseline length, angular field of view, data volume and dynamic range. The SKA will be larger by orders of magnitude on all these aspects \cite{ska-phase1-2013}, and will therefore require new advances in solving the imaging problem. I will give an overview of the challenges involved, and summarize the currently available approaches.

\section{W-term correction}
The first step in imaging is the mathematical inversion of the visibility function -- the function that maps the sky brightness to the complex correlation products. For simple arrays, this function simplifies into a two-dimensional Fourier transform, but arrays such as the SKA that will have a wide field of view and non-coplanar elements require correction for the '$w$-terms', which describe the distance of an element to the plane. Larger $w$-terms increase the cost of the inversion. Two algorithms to correct for the $w$-terms are $w$-projection \cite{wprojection-cornwell} and $w$-stacking \cite{ska-memo-regridding-2011}. With common imaging parameters, these algorithms have a squared and linear dependency on the size of the $w$-term respectively, but the faster $w$-stacking algorithm requires more memory \cite{offringa-wsclean-2014}. Further improvement can be made by combining either of these techniques with snapshot imaging. This process, called $w$-snapshot imaging, decreases the size of the $w$-term, but requires regridding of the snapshot images before they can be integrated or deconvolved \cite{widefield-imaging-ska-cornwell}. An implementation of this algorithm in \textsc{wsclean} has shown that this is effective only for very off-zenith imaging or for imaging very large fields of view \cite{offringa-wsclean-2014}.

\section{Correction of directional effects}
Direction-dependent effects (DDEs) that change over time, such as a changing primary beam or ionosphere -- which will be the case for the SKA -- have to be corrected for before images from different epochs are added together. Pathfinders have shown this can be corrected during the inversion step using the $a$-projection technique \cite{awimager-2013,aprojection-2008} or, when the changes are uniform over the elements, by correcting snapshots in image space \cite{offringa-wsclean-2014,mwa-interferometric-imaging}. A hybrid of these two can be used to reduce the cost.

\section{Deconvolution}

\section{Parallellism}

\section{\textsc{wsclean}}
WSClean is a new and fast widefield imager developed in particular for imaging MWA data, but it is a generic imager able to deconvolve, produce beam-corrected images for fixed tile arrays and perform wideband deconvolution. Since recently it also has a novel multi-resolution multi-scale clean algorithm \cite{offringa-wsclean-2014}.

\DeclareRobustCommand{\TUSSEN}[3]{#3}

\bibliographystyle{IEEEtran}
\bibliography{bib}

\end{document}


