\documentclass[a4paper,11pt]{article} 
%\usepackage[style=nature,backend=biber]{biblatex}
\usepackage{graphicx}
\usepackage[font={it},labelfont={bf}]{caption}
\usepackage{times}
\usepackage{color}
%\usepackage{fixltx2e} % fixes placing of image positions
\usepackage{authblk}
\usepackage{amsmath}
\usepackage{courier}

\definecolor{customhdrcolor}{rgb}{0.0,0.0,0.0}
%\definecolor{customcitecolor}{rgb}{0.0,0.25,0.25}
\definecolor{customcitecolor}{rgb}{0.0,0.5,0.75}
%\definecolor{customlinkcolor}{rgb}{0.0,0.0,1.0}
\definecolor{customlinkcolor}{rgb}{0.0,0.5,0.75}

\usepackage[colorlinks=true,linkcolor=customlinkcolor,urlcolor=customlinkcolor,citecolor=customcitecolor,pdftex]{hyperref}

\ifpdf\pdfinfo{/Title      (Memo: MWA beam-correction formulae)
               /Author     (A. R. Offringa et al.)
               /Keywords   (MWA, beam)
        }
\else\usepackage{graphics}\fi

\setlength{\pdfpageheight}{\paperheight}
\setlength{\pdfpagewidth}{\paperwidth}

\title{Memo: MWA beam-correction formulae}

\author{A.~R.~Offringa}
%\author[1,2]{A.~R.~Offringa}
%\affil[1]{RSAA, Australian National University, Mt Stromlo Observatory, via Cotter Road, Weston, ACT 2611, Australia}
%\affil[2]{ARC Centre of Excellence for All-Sky Astrophysics (CAASTRO)}
\begin{document}

\label{firstpage}
\maketitle

\section{Analytical calculation of voltage gain}
The analytical gain of a tile is calculated from the analytical gain of a dipole with ground screen, multiplied by a factor $\eta$ that represents how coherent the dipoles add together. The (complex) tile factor $\eta$ is calculated with:\footnote{Taken from the RTS (2013-08-13), file \texttt{InstrumentalCalibration.c} written by D. Mitchell}
\begin{equation}
 \eta = \frac{1}{16}\sum\limits_{a=1}^{16} e^{\frac{-2 \pi i}{\lambda} \left(\Delta x_a \sin \mathcal{Z} \sin \chi + \Delta y_a \sin \mathcal{Z} \cos \chi + \Delta z_a \cos \mathcal{Z} - \Delta \tau_a \right) },
\end{equation}
where $\lambda$ is the wavelength, $\Delta x_a, \Delta y_a$ and $\Delta z_a$ are the relative positions of dipole $a$ in the tile, $\mathcal{Z}$ is zenith angle, $\chi$ is the azimuth and $\Delta \tau_a$ is the delay applied on dipole $a$ by the beam former. The $2 \times 2$ matrix $\textbf{E}$ describing the electrical gains of an individual dipole\footnote{Also described here: http://mwa-lfd.haystack.mit.edu/twiki/bin/view/Main/PrimaryBeamModels by D. Mitchell} is calculated with:
\begin{equation}
\mathbf{E} = \frac{2 \sin \left( \frac{2 \pi h}{\lambda} \cos \mathcal{Z}\right)}{2 \sin \left( \frac{2 \pi h}{\lambda}\right)}
\left( \begin{array}{ccc}
\cos l \cos \delta + \sin l \sin \delta \cos \textrm{ha} & ; & -\sin l \sin \textrm{ha} \\
\sin \delta \sin \textrm{ha} & ; & \cos \textrm{ha}
\end{array} \right).
\end{equation}
Here, $l$ is latitude, $\delta$ is declanation, $\textrm{ha}$ is hour angle and $h$ is the height of tile. The factors of two cancel out, but are left there because the denominator now forms the normalization towards zenith. Finally, the Jones gain matrix $B$ of the full tile is calculated by multiplying the dipole matrix with the tile factor,
\begin{equation}
\mathbf{B} = \eta \mathbf{E}.
\end{equation}
\section{Applying beam gains in image space}
Given the $2\times 2$ complex Jones matrix $\mathbf{B}$ that represents the voltage gain of the beam in a certain direction, the RTS corrects a polarized pixel, given by the vector $\textbf{v} = \left(v_\textrm{xx}, \textrm{real}(v_\textrm{xy}), \textrm{imag}(v_\textrm{xy}), v_\textrm{yy} \right)^T$, for that direction in the image plane to a beam-corrected vector containing the Stokes values with the following calculation:\footnote{Taken from the RTS (2013-08-13), file \texttt{pb\_correct.c} written by S. Ord}
\begin{equation} \label{apply-beam-image-space}
 \textbf{s} =
\mathbf{T} \left[ \left(\mathbf{B} \otimes \mathbf{B}^* \right) \left(\mathbf{B} \otimes \mathbf{B}^* \right)^*\right]^{-1} \mathbf{Y} \textbf{v},
\end{equation}
with
\begin{equation}
\textbf{s} = \left( \begin{array}{c}
s_\textrm{\tiny{I}} \\
s_\textrm{\tiny{Q}} \\
s_\textrm{\tiny{U}} \\
s_\textrm{\tiny{V}} \end{array} \right),
T = \left( \begin{array}{cccc}
1 & 0 & 0 & 1 \\
1 & 0 & 0 & -1 \\
0 & 1 & 1 & 0 \\
0 & -i & -i & 0 \end{array} \right), \mathbf{Y} = \left( \begin{array}{cccc}
1 & 0 & 0 & 0 \\
0 & 1 & i & 0 \\
0 & 1 & -i & 0 \\
0 & 0 & 0 & 1
\end{array} \right),
\end{equation}
where $\otimes$ the Kronecker product and $\mathbf{A}^*$ the conjugate transpose of matrix $\mathbf{A}$. The output $\textbf{s}$ is a vector with the Stokes parameter. The matrix $\mathbf{T}$ converts from instrumental basis (xx, xy, yx, yy) to the Stokes parameter and hence is applied last.

Matrix $\mathbf{Y}$ is applied because the RTS does not store both the xy and yx images, but instead stores one complex xy image. Its real values are stored in the 2nd index, and the imaginary values are stored in the 3rd index of vector $\textbf{v}$. Multiplying $\mathbf{v}$ with $\mathbf{Y}$ therefore has the following effect: $\mathbf{Y}\textbf{v} = \left(v_\textrm{xx},v_\textrm{xy}, \overline{v_\textrm{xy}},v_\textrm{yy}\right)^T$. $\mathbf{Y}$ can be left out if the input is alread xx, xy, yx, yy.

\section{Applying the beam more efficiently}
As a side note, Eq.~\eqref{apply-beam-image-space} involves a $4\times 4$ complex matrix inverse, which can be computationally done more efficiently by decomposing it in two $2\times 2$ complex matrix inversions:
\begin{equation} \notag
 \textbf{s} =
\mathbf{T} \left[ \left(\mathbf{B} \otimes \mathbf{B}^* \right) \left(\mathbf{B} \otimes \mathbf{B}^* \right)^*\right]^{-1} \mathbf{Y} \textbf{v}
\end{equation}
\{ Conjugate transpose is distributive over the Kronecker product  \}
\begin{equation} \notag
 \textbf{s} =
\mathbf{T} \left[ \left(\mathbf{B} \otimes \mathbf{B}^* \right) \left(\mathbf{B}^* \otimes \mathbf{B} \right)\right]^{-1} \mathbf{Y} \textbf{v}
\end{equation}
\{ Using associativity of K.p., i.e., $\left(\mathbf{A} \otimes \mathbf{B} \right) \left(\mathbf{C} \otimes \mathbf{D} \right) = (\mathbf{AC}) \otimes (\mathbf{BD})$ \}
\begin{equation} \notag
 \textbf{s} = \mathbf{T} \left[ (\mathbf{BB}^*) \otimes (\mathbf{B}^*\mathbf{B}) \right]^{-1} \mathbf{Y} \textbf{v}
\end{equation}
\{ Inverse is distributive over the Kronecker product \}
\begin{equation}
 \textbf{s} = \mathbf{T} \left[ (\mathbf{BB}^*)^{-1} \otimes (\mathbf{B}^*\mathbf{B})^{-1} \right] \mathbf{Y} \textbf{v}
\end{equation}
Note that $\mathbf{BB}^*$ and $\mathbf{B}^*\mathbf{B}$ are not equal (unless $\mathbf{B}$ is unitary, which it is not in general), but they are $2\times 2$ matrices, and their inverse thus has an analytical solution. This is much faster and easier than using a general library to do a $4 \times 4$ inversion.
\label{lastpage}

\end{document}
